\documentclass[11pt,letterpaper,fleqn]{article}
\usepackage{rotating}
\usepackage{graphicx}
\usepackage{amsfonts,amsmath}
\usepackage{subfigure} % need for subfigures
\usepackage{wrapfig}
\usepackage[margin=1in]{geometry}
\usepackage{cite}
\usepackage{amssymb}
\usepackage[final]{pdfpages}
\usepackage{hyperref,enumitem}
\hypersetup{
colorlinks, citecolor=blue, filecolor=black, linkcolor=blue, urlcolor=blue}
\usepackage{booktabs}
\usepackage{multirow}
\usepackage{siunitx}
\sisetup{group-separator = {,}}
\usepackage{titlesec}
\titlespacing\section{0pt}{6pt plus 4pt minus 2pt}{0pt plus 0pt minus 0pt}
\titlespacing\subsection{0pt}{6pt plus 4pt minus 2pt}{0pt plus 0pt minus 0pt}
\titlespacing\subsubsection{0pt}{6pt plus 4pt minus 2pt}{0pt plus 0pt minus 0pt}
\usepackage{titling} % Allows custom title configuration
\usepackage{fancyhdr} % Needed to define custom headers/footers
\usepackage{lastpage} % Used to determine the number of pages in the document
\usepackage[english]{babel}
\usepackage{blindtext}
\usepackage{draftwatermark}
\SetWatermarkText{Draft v0.0}

% Add you own definitions here (file report-defs.sty).
\usepackage{report-defs}

%-------------------------------------------------------------------------------
%   TITLE SECTION
%-------------------------------------------------------------------------------

\date{} % blank out date from fancyhdr

\pretitle{\vspace{-50pt} \begin{flushleft} \fontsize{15}{15} \usefont{OT1}{phv}{b}{n} \selectfont} % Horizontal rule before the title
\title{\large Summary of Blueprint Workshop: \\
\vspace{1pt}
\color{red}
\LARGE \textit{Analysis Systems R\&D on Scalable Platforms} \\
\color{black} \normalsize
\vspace{10pt}
June 21--22, 2019 \\
New York University \\
Meeting URL: \href{https://indico.cern.ch/event/820946/}{https://indico.cern.ch/event/820946/}
} % Your article title
\posttitle{\par\end{flushleft}\vskip 0.5em} % Whitespace under the title
\preauthor
{
  \begin{flushleft}
    \vspace{-13pt}
    {\normalfont Workshop Organizers:} \\
    \vspace{3pt}
    \large \lineskip 0.5em \usefont{OT1}{phv}{b}{sl} \color{black}} % Author font configuration
    \author{Kyle Cranmer {\normalfont(New York University)}
      \and  Rob Gardner {\normalfont(University of Chicago)}
      \and  Mark Neubauer {\normalfont(University of Illinois at Urbana-Champaign)}
      }
      \postauthor{\par
      \vspace{10pt}
      {\normalfont Summary prepared by:} \\
      \vspace{3pt}
      Mark Neubauer {\normalfont(University of Illinois at Urbana-Champaign)}

\end{flushleft} \vspace{-10pt} \HorRule} % Horizontal rule after the title

\renewcommand{\and}{\\}

%-------------------------------------------------------------------------------

\begin{document}
\maketitle % Print the title
\normalfont

\thispagestyle{firststyle}

\vspace{-45pt}

\section*{Major Goals}
\vspace{3pt}
\begin{itemize}
  \item Review the status of AS milestones and deliverables
  \item Develop the Scalable Systems Laboratory (SSL) scope, architecture and plans, using Analysis Systems (AS) R\&D activties as concrete examples.
  \item Develop requirements on SSL to support the AS area, particularly the prototyping, benchmarking and scaling of AS deliverables toward deployment.
  \item Increase the visibility of SSL and AS R\&D beyond IRIS-HEP to facilitate partnerships with organizations that could potentially provide software and computing resources for SSL.
  \item Get informed on latest developments in technologies and methods relevant for SSL and AS.
\end{itemize}

\section*{Key Outcomes}
\vspace{3pt}
\begin{itemize}
  \item Communication of the AS area plans and preliminary requirements to SSL.
  \item Kubernetes as a planned "common denominator" for SSL, increasing the capabilities through flexible infrastruture. This idea spawned plans for a multi-site SSL "substrate" project.
  \item Productive engagement of the AS/SSL team with representatives from NCSA, SDSC, NYU Research Computing, industry \& cloud providers (Google, Redhat), generating action items.
\end{itemize}
\vspace{-5pt}
~~~~Planning around
\vspace{-5pt}
\begin{itemize}
  \item collection and curatation of analysis use cases, each with a reference implementation.
  \item translation of analysis examples into new specifications, providing feedback and iteration.
  \item development of initial specifications for user-facing interfaces to analysis system components.
  \item benchmarking of existing \& prototype AS components and tying into SSL infrastructure.
  \item development of accelerator-based fitting \& statistical tools (and other relevant components).
  \item integrating prototypes of AS components into SSL, followed by benchmarking \& assessment.
\end{itemize}

%-------------------------------------------------------------------------------

\newpage
\thispagestyle{reststyle}

\section{Overview}
Together with the OSG-LHC, the Scalable Systems Laboratory (SSL) is designed to be the primary integration path to deliver the output of IRIS-HEP R\&D activities into the distributed and scientific production infrastructure of the experiments. The aim of this workshop is to further develop the IRIS-HEP SSL concept using specific R\&D examples from the AS area, including  low-latency, query-based data systems and modular, reusable cyberinfrastructure for physics inference and results dissemination. Registered attendees include those from IRIS-HEP (primarily SSL and AS areas), US ATLAS/CMS operations programs, national labs, CERN, supercomputing centers (SDSC, NCSA), university research IT, and industry (RedHat, Google).

The venue for the workshop was the Physics Department at New York University and was hosted by Kyle Cranmer.

\section{Attendees}
There were 26 \href{https://indico.cern.ch/event/820946/registrations/participants}{registered participants} for workshop, with all but a few attending in person. The workshop attendees were:
Andrew Chien (Chicago),
Andrew Melo (Vanderbilt),
Aravindh Puthiyaparambil (Red Hat),
Benjamin Galewsky (Illinois/NCSA),
Dan S. Katz (Illinois/NCSA),
David Ackerman (NYU),
Edgar Fajardo (SDSC),
Eric Borenstein (NYU),
Gordon Watts (Washington),
Ianna Osborne (Fermilab),
Jim Pivarski (Princeton),
Kyle Cranmer (NYU),
Lincoln Bryant (Chicago),
Lindsey Gray (Fermilab),
Mark Neubauer (Illinois),
Mason Proffitt (Washington),
Matthew Feickert (SMU),
Nils Krumnack (Iowa State),
Ricardo Brito Da Rocha (CERN),
Rob Gardner (Chicago),
Sanjay Arora (Red Hat),
Stephen Fang (Google),
Stratos Efstathiadis (NYU),
Tatiana Polunina (NYU),
Tim Boerner (Illinois/NCSA),
Wei Yang (SLAC)

\section{Goals}
The primary goals of the workshop were to
\begin{itemize}
  \item Review the status of AS milestones and deliverables
  \item Develop the Scalable Systems Laboratory (SSL) scope, architecture and plans, using Analysis Systems (AS) R\&D activties as concrete examples.
  \item Develop requirements on SSL to support the AS area, particularly the prototyping, benchmarking and scaling of AS deliverables toward deployment.
  \item Increase the visibility of SSL and AS R\&D beyond IRIS-HEP to facilitate partnerships with organizations that could potentially provide software and computing resources for SSL.
  \item Get informed on latest developments in technologies and methods relevant for SSL and AS.
\end{itemize}

\section{Activites}
\blindtext[2]

\subsection{Presentations and Discussion}
\blindtext[2]

\subsection{Breakout Session}
\blindtext[2]

\section{Key Outcomes}
The workshop lead to several key outcomes, including:

\begin{itemize}
  \item Communication of the AS area plans and preliminary requirements to SSL.
  \item Kubernetes as a planned "common denominator" for SSL, increasing the capabilities through flexible infrastruture. This idea spawned plans for a multi-site SSL "substrate" project.
  \item Productive engagement of the AS/SSL team with representatives from NCSA, SDSC, NYU Research Computing, industry \& cloud providers (Google, Redhat), generating action items.
\end{itemize}
\vspace{-5pt}
~~~~Planning around
\vspace{-5pt}
\begin{itemize}
  \item collection and curatation of analysis use cases, each with a reference implementation.
  \item translation of analysis examples into new specifications, providing feedback and iteration.
  \item development of initial specifications for user-facing interfaces to analysis system components.
  \item benchmarking of existing \& prototype AS components and tying into SSL infrastructure.
  \item development of accelerator-based fitting \& statistical tools (and other relevant components).
  \item integrating prototypes of AS components into SSL, followed by benchmarking \& assessment.
\end{itemize}

\section{Action Items}
\blindtext[2]

\section{Feedback from Attendees}
\begin{itemize}
  \item Some preparatory documents
  \item Back to Ben’s point about IRIS needing community developement / management etc.
  \item Start planning earlier (obvs)
  \item AS contribution to accelerated inference blueprint
\end{itemize}

\section{Summary}
\blindtext[2]

\appendix
\newpage
\section{Revision History}

\vspace{8pt}
\begin{itemize}
  \item Version 0.0
  \vspace{-5pt}
  \begin{itemize}
    \item Initial version
  \end{itemize}
\end{itemize}

\end{document}
