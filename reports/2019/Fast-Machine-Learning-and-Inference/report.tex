\documentclass[11pt,letterpaper,fleqn]{article}
\usepackage{rotating}
\usepackage{graphicx}
\usepackage{amsfonts,amsmath}
\usepackage{subfigure} % need for subfigures
\usepackage{wrapfig}
\usepackage[margin=1in]{geometry}
\usepackage{cite}
\usepackage{amssymb}
\usepackage[final]{pdfpages}
\usepackage{hyperref,enumitem}
\hypersetup{
colorlinks, citecolor=blue, filecolor=black, linkcolor=blue, urlcolor=blue}
\usepackage{booktabs}
\usepackage{multirow}
\usepackage{siunitx}
\sisetup{group-separator = {,}}
\usepackage{titlesec}
\titlespacing\section{0pt}{6pt plus 4pt minus 2pt}{0pt plus 0pt minus 0pt}
\titlespacing\subsection{0pt}{6pt plus 4pt minus 2pt}{0pt plus 0pt minus 0pt}
\titlespacing\subsubsection{0pt}{6pt plus 4pt minus 2pt}{0pt plus 0pt minus 0pt}
\usepackage{titling} % Allows custom title configuration
\usepackage{fancyhdr} % Needed to define custom headers/footers
\usepackage{lastpage} % Used to determine the number of pages in the document
\usepackage[english]{babel}
\usepackage{blindtext}
%\usepackage{draftwatermark}
%\SetWatermarkText{Draft v0.0}

% Add you own definitions here (file report-defs.sty).
\usepackage{report-defs}
%\setlength{\headheight}{130pt}% ...at least 51.60004pt
\usepackage{tikzpagenodes}

%-------------------------------------------------------------------------------
%   TITLE SECTION
%-------------------------------------------------------------------------------

\date{} % blank out date from fancyhdr

\pretitle{
\vspace{-50pt}
\begin{flushleft} \fontsize{15}{15} \usefont{OT1}{phv}{b}{n} \selectfont} % Horizontal rule before the title
\title{\large Summary (Draft v0.0) of Blueprint Workshop: \\
\vspace{1pt}
%\color{red}
\LARGE \textit{Fast Machine Learning and Inference} \\
\color{black} \normalsize
\vspace{10pt}
September 10--11, 2019 \\
Fermilab \\
Meeting URL: \href{https://indico.cern.ch/event/822126}{https://indico.cern.ch/event/822126}
} % Your article title
\posttitle{\par\end{flushleft}\vskip 0.5em} % Whitespace under the title
\preauthor
{
  \begin{flushleft}
    \vspace{-13pt}
    {\normalfont Workshop Organizers:} \\
    \vspace{3pt}
    \large \lineskip 0.5em \usefont{OT1}{phv}{b}{sl} \color{black}} % Author font configuration
    \author{Kyle Cranmer {\normalfont(New York University)}
      \and  Rob Gardner {\normalfont(University of Chicago)}
      \and  Mark Neubauer {\normalfont(University of Illinois at Urbana-Champaign)}
      }
      \postauthor{\par
      \vspace{10pt}
      {\normalfont Summary prepared by:} \\
      \vspace{3pt}
      Mark Neubauer {\normalfont(University of Illinois at Urbana-Champaign)}
      \and  Rob Gardner {\normalfont(University of Chicago)}

\end{flushleft} \vspace{-10pt} \HorRule} % Horizontal rule after the title

\renewcommand{\and}{\\}

%-------------------------------------------------------------------------------

\begin{document}
\maketitle % Print the title
\normalfont

\thispagestyle{firststyle}

\vspace{-250pt}
\hspace{360pt}
\includegraphics[height=30mm]{../../../figures/iris-hep-bluprint-logo.png}

\vspace{120pt}
\section*{Major Goals}
\vspace{3pt}
\begin{itemize}
  \item Review the status of the Analysis Systems (AS) milestones and deliverables to inform the needs for a collaborative development and testing platform.
  \item Develop the Scalable Systems Laboratory (SSL) architecture and plans, using AS R\&D activities as specific examples.
  \item Develop requirements on SSL to support the AS area, particularly the prototyping, benchmarking and scaling of AS deliverables toward production deployment.
  \item Increase the visibility of SSL and AS beyond IRIS-HEP to facilitate partnerships with organizations that might provide software and computing resources toward these objectives.
  \item Get informed on latest developments in open source technologies and methods important for the success of the SSL and AS R\&D areas of the Institute.
\end{itemize}

\section*{Key Outcomes}
\vspace{3pt}
\begin{itemize}
  \item Communication of the AS area plans leading to a set of requirements to SSL team.
  \item Kubernetes identified as a planned {\it common denominator} technology for the SSL, increasing our innovation capability through flexible infrastructure.
  \item Plans for a multi-site SSL {\it substrate project} that will federate SSL contributions from multiple resource providers (institutes and public cloud), offering the AS area a flexible platform for service deployment at scales needed to test the viability of system designs.
  \item Productive engagement of the AS/SSL team with representatives from NCSA, SDSC, NYU Research Computing, industry \& cloud providers (Google, Redhat), generating actions and informing Year 2 planning of IRIS-HEP.
  \item A vision for an SSL that serves as an innovation space for AS developers and a testbed to prototype next generation infrastructure patterns for future HEP computing environments.
\end{itemize}

%-------------------------------------------------------------------------------

\newpage
\pagestyle{reststyle}

\section{Overview}
\vspace{0.2cm}
Together with the ...

\appendix
\newpage
\section{Revision History}

\vspace{8pt}
\begin{itemize}
  \item Version 0.0
  \vspace{-5pt}
  \begin{itemize}
    \item Initial version
  \end{itemize}
\end{itemize}

\end{document}
